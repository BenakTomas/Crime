\chapter{Requirements and Goals}
The Crime statistics dataset would become a new member of the \textit{linked.opendata.cz} datasets family.
It is intended to be served on a \textit{linked.opendata.cz} server and thus residing in the \textit{linked.opendata.cz} URL namespace.

\section{Requirements}
Several requirements have been proposed:\\
\begin{itemize}
	\item[information detail and completeness]
	The published dataset is intended to be the most detailed and complete source of public crime statistics data there is.
	Of course some data cannot be made available for public, typically personal data.
	
	\item[ability to stay up-to-date]
	There should be an easy-to-use mechanism to perform the dataset update when new data becomes available.
	
	\item[public]
	The dataset should be made publicly available through standard Linked Data technologies. This means it should be discoverable and its data should be made accessible for querying.
	
	\item[linking to other datasets]
	The dataset would be connected to other datasets where such links emerge naturally. These connections enable the Crime dataset to become a part of the global data-space, the Web of Data.
	
	\item[extensibility]
	The Crime Ontology (vocabulary) developed to describe the Crime dataset structure should be easily extensible when the source data structure changes.
	
	\item[usefulness]
	The dataset should prove itself useful.
	
\end{itemize}


\section{Goals}
Building upon the requirements we set the goals of this thesis:

\begin{itemize}
	\item[the data source] Gather the most complete and detailed crime statistics data.
	Exclude the personal data.
	
	\item Publish the gathered data as an RDF dataset in an RDF store and make its data available at a SPARQL endpoint.
	
	\item Develop an ontology or a vocabulary to describe and maintain the structure and semantics of the published data. The ontology will be built reusing the existing well-known vocabularies to make it easier to comprehend, use and extend.
	
	\item[update mechanism] An update mechanism will be developed to make it easy for the dataset maintainer to perform the dataset update in case new data is available.
	
	\item The dataset will be linked with resources in other datasets, for example DBPedia and linked.opendata.cz datasets.
	
	\item[usefulness] An Android demo-application will be created to demostrate the usability and usefulness of the dataset. The application will provide crime statistics based on the user's current location.
\end{itemize}


\section{Potential Users}
There are three groups of potential users of both the dataset and the demo-application.
\begin{itemize}
	\item[application developers] The primary target audience of the Crime dataset. The developers can issue interesting queries on the data, mashing them up with related data using the geography location as the key.
	
	\item[dataset and ontology creators]
	The other datasets creators may want to reuse a part of the Crime vocabulary when talking about crime.
	
	\item[Android users] Every user of a device powered by a supported version of Android is a potential user of the demo-application. The main value for such user would probably be entertainment.
\end{itemize}
