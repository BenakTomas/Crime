%%% Kapitola o technologiích a software - sémantický web, geodata, otevřená data v kontextu obou zmíněných oblastí

\chapter{Koncepty, technologie, software}

V předchozí kapitole jsme se seznámili s konceptem otevřených dat, poznali jejich výhody i rizika spojená s jejich publikací, a také nahlédli na konkrétní příklady otevřených datových sad a nad nimi vybudovaných aplikací.
Vhodnou platformou pro otevřená data se ukázal být tzv. sémantický web: (postupně koncepty, technologie, software)
\begin{itemize}
	\item sémantický web vs web dokumentů
	\item Linked data: principy, vztah k otevřeným datům
	\item RDF jako jazyk sémantického webu a Linked data, serializace RDF
	\item ontologie (RDFS, SKOS, DCTERMS, VOID, ...)
	\item publikace, SPARQL, RDF Store
	\item geodata: GML, shapefile, open geodata, RDF a geodata, SPARQL a geodata
	\item SW nástroje: Datalift, D2RQ, SILK, Virtuoso, OpenSesame, Jena, Tomcat, PostGIS
\end{itemize}
