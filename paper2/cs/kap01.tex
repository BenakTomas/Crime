\chapter{Otevřená data}
\section{Co jsou otevřená data?}
Otevřená data jsou poměrně širokým pojmem, pro který existuje celá řada obecných i specializovaných definic. Z těch obecných zmiňmě hojně používanou definici Open definition\footnote{http://opendefinition.org/od/2.1/en/} a definici Tima Bernerse-Lee, která je konkrétnější a definuje stupně otevřenosti dat.\footnote{5stardata.info} Specializované definice otevřených dat z těch obecných vycházejí, uvádějí ale konkrétnější požadavky a soustřeďují se na ty aspekty, které jsou relevantní pro cílovou doménu, kde mají být data použita (např. právě veřejný sektor). Pro potřeby této práce si i my definici otevřených dat přizpůsobíme. Otevřenými daty budeme rozumět data, která
\begin{itemize}
	\item může kdokoli používat, upravovat a dále šířit
	\item jsou dostupná a (snadno) dohledatelná na Internetu
	\item jsou dostupná ve formátu, jehož specifikace je volně přístupná
	\item jsou strukturovaná způsobem, který umožňuje je strojově zpracovávat
\end{itemize}

% Trochu více detailu o vhodných licencích.
První bod definice je jádrem Open definition a je asi tím nejdůležitějším, protože umožňuje bezproblémové a volné užívání a šíření dat. Občas se může vyskytnout potřeba nějak způsob nakládání s daty upravit, například zavést povinnost při každém jejich použití uvést autora. Pro tyto účely lze užít některé ze sady specializovaných licencí, např. licence z populární sady Creative Commons, jako Creative Commons 3.0\footnote{https://creativecommons.org/licenses/by/3.0/cz/} nebo Public domain zero\footnote{https://creativecommons.org/publicdomain/zero/1.0/}.

Požadavek na formát s volně přístupnou specifikací vyplývá ze snahy vyhnout se závislosti na software, za jehož užití pro práci s otevřenými daty by musel uživatel platit. Z tohoto pohledu nejsou formáty jako XLS nebo DOC firmy Microsoft otevřené, protože jejich použití je vázáno na nutnost zakoupit příslušné programy sady Microsoft Office. Naproti tomu s formáty jako CSV, XML, RDF, nebo otevřenými verzemi Microsoftích formátů DOCX a XLSX je možné pracovat pomocí volně (zadarmo) dostupných nástrojů.

Vnitřní struktura dat hraje roli pro možnost data strojově zpracovávat. Ačkoliv člověk je v textu schopen pohledem různým částem dokumentu (nadpis, poznámka pod čarou, seznam) přiřadit význam a odpovídajícím způsobem je interpretovat, stroj (software) toho schopen není (\footnote{Výjimkou je software zpracovávající přirozený jazyk (např. pomocí metod strojového učení), který je ale mimo kontext této práce a jehož rozpoznávací schopnosti nejsou stoprocentní.} Tento problém se řeší zavedením určitého vnitřního řádu, který diktuje vnitřní strukturu dat. Taková struktura může být poměrně volná, jako u formátu CSV, kde jsou data umístěna v řádcích a jednotlivá pole v řádku jsou oddělená čárkou. Naopak různé binární formáty\footnote{https://en.wikipedia.org/wiki/Binary\_file} vyžadují strukturu pevnou, kdy jsou data pevně zarovnávána do bloků začínajících na pevně daných adresách v paměti, což umožňuje zpracovávajícímu software spolehnout se, že data s určitým významem jsou v paměti počítače uložena na předem definovaném místě. Možností strukturování dat, která je někde napůl cesty mezi prvními dvěma popsanými způsoby, je značkování. Například formát XML používá tzv. tagy, neboli textové značky, které ohraničují část textu a dávají mu tak význam, který umí aplikace interpretovat. Takto například  z původní, pro program nečitelné věty \textit{\uv{Jmenuji se Tomáš Beňák.}}, lze pomocí XML vyrobit větu \textit{\uv{Jmenuji se <Jmeno>Tomáš</Jmeno><Prijmeni>Beňák</Prijmeni>.}}, ze které už program, který v textu výskyt tagů \textsf{<Jmeno>} a \textsf{<Prijmeni>} očekává, může extrahovat jméno a příjmení.

\section{Zdroje otevřených dat}

Přirozeným zdrojem otevřených dat jsou instituce veřejné správy a samosprávy, jako například vláda, ministerstva, obce a úřady, obecně pak veřejný sektor, zahrnující různé další nevládní a neziskové organizace. Na řadu z informací poskytovaných orgány veřejné moci existuje zákonný nárok, například v České republice vyplývající ze zákona 106/1999 Sb. o svobodném přístupu k informacím, a zákona 123/1998 Sb. o právu na informace o životním prostředí. Kromě toho jsou členské státy Evropské unie vázány směrnicí PSI\footnote{Směrnice Evropského parlamentu a Rady č. 2003/98/ES} o opakovaném použití informací veřejného sektoru, která ukládá členským zemím povinnost poskytovat veřejná data jako otevřená.

Data poskytovaná veřejnou správou a samosprávou typicky zahrnují např. informace o výdajích státu, veřejných rozpočtech, výběrových řízeních na veřejné zakázky, výsledky voleb, přehled legislativy nebo základní statistické ukazatele pro danou krajinu. Open knowledge foundation zveřejnila v rámci projektu Open Data Index\footnote{http://index.okfn.org} seznam datových sad veřejného sektoru doporučených ke zveřejnění.

Zveřejňování dat veřejné správy bývá zastřešeno na úrovni státu portály pro otevřená data, které poskytují katalogy otevřených datových sad, poskytovaných institucemi veřejné správy dané země. V takových katalozích vydavatelé dat registrují poskytovaná data a uživatelé katalogů v nich pak otevřené datové sady na základě různých kritérií vyhledávají. V zahraničí jsou ukázkovými příklady takových projektů portály otevřených dat USA a Velké Británie{https://www.data.gov resp. https://data.gov.uk}. Otevřené datové sady veřejných institucí v České republice jsou katalogizovány v Národním katalogu otevřených dat (NKOD)\footnote{http://portal.gov.cz/portal/obcan/rejstriky/data/97898} v rámci portálu veřejné správy\footnote{http://portal.gov.cz}. Výhodou centralizovaného řešení je zajištění jednotných standardů pro publikaci a katalogizaci dat, a také nutná metodická podpora vydavatelům dat.

Kromě státních institucí otevřená data zveřejňují i různé nestátní, soukromé, mezinárodní, nevládní, akademické a jiné neziskové nebo obecně prospěšné organizace, zpravidla na vlastních webech a ve vlastních katalozích. Příkladem takové organizací v České republice jsou Opendata.cz, v zahraničí otevřená data zveřejňuje britský deník Guardian\footnote{http://www.theguardian.com/data}, nebo mezinárodní organizace jako OSN\footnote{http://data.un.org} nebo OECD\footnote{http://stats.oecd.org}.

\section{Existující datové sady}

V současnosti již existuje celá řada otevřených datových sad. Klíčovým faktorem pro úspěch otevřených dat je zapojení a podpora státních institucí. Státní instituce jednak slouží jako poskytovatel (surových) dat, ale často také poskytují pro otevřená data platformu pro jejich publikaci a katalogizaci na celostátní úrovni. Příkladem takového (úspěšného) úsilí jsou portály pro otevřená data vlád USA nebo Velké Británie\footnote{https://www.data.gov, https://data.gov.uk}. V České republice takovou platformu poskytuje Ministerstvo vnitra v rámci projektu opendata.gov.cz\footnote{http://opendata.gov.cz} na portálu veřejné správy\footnote{http://portal.gov.cz}. Tato platforma je zaměřena primárně na orgány veřejné správy, data jsou katalogizována v Národním katalogu otevřených dat (NKOD).

Kromě státních institucí otevřená data produkují také jiné subjekty, jako jsou různé nevládní a neziskové, ale i komerční organizace. V České republice je jednou z takovýchto nezisková organizace Opendata.cz, která na svých stránkách\footnote{http://linked.opendata.cz/en} poskytuje přístup k vlastním datasetům.

Nad otevřenými daty dosud vzniklo mnoho zajímavých a užitečných aplikací jak ve světě, tak i v ČR. Z těch zahraničních jmenujme například TheyWorkForYou\footnote{http://www.TheyWorkForYou.com}, zaznamenávající hlasování členů britského parlamentu, WhereDoesMyMoneyGo \footnote{http://www.wheredoesmymoneygo.org/}, poskytující přehled o využití daní britských daňových poplatníků, nebo celé sady aplikací od MySociety\footnote{https://www.mysociety.org/projects/} a od Sunlight Foundation\footnote{http://sunlightfoundation.com/tools/}. V České republice máme

[Přínosy a rizika]

[Otevřená data v ČR - instituce, komunita, zdroje informací (Jak otevírat data), poskytovatelé otevřených dat: veřejný sektor (ministerstva, samosprávné celky, vládní orgány, nevládky a neziskovky), soukromý sektor.
Zmínit konkrétní instituce: Fond Otakara Motejla a Fórum otevřených dat, MV ČR, NKOD, portal.gov.cz, opendata.cz]