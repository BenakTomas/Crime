\chapter{Otevřená data}
\section{Co jsou otevřená data?}
% TODO tady uvést příklady otevřených dat - přehledy hospodaření, výsledky voleb, základní statistiky.
Otevřená data jsou poměrně širokým pojmem, pro který existuje celá řada obecných i specializovaných definic. Z těch obecných zmiňme hojně používanou definici Open definition\footnote{http://opendefinition.org/od/2.1/en/} a definici Tima Bernerse-Lee, která otevřená data klasifikuje podle sady kritérií definujících \uv{otevřenost} dat.\footnote{5stardata.info} Specializované definice otevřených dat pak z těch obecných vycházejí, uvádějí ale konkrétnější požadavky a soustřeďují se na ty aspekty, které jsou relevantní pro doménu, kde mají být data použita. Pro potřeby této práce si i my definici otevřených dat přizpůsobíme a otevřenými daty budeme rozumět data, která
\begin{itemize}
	\item může kdokoli používat, upravovat a dále šířit
	\item jsou dostupná a (snadno) dohledatelná na Internetu
	\item jsou dostupná ve formátu, jehož specifikace je volně přístupná
	\item jsou strukturovaná způsobem, který umožňuje je strojově zpracovávat
\end{itemize}

% Trochu více detailu o vhodných licencích.
První bod definice je jádrem Open definition a je tím nejdůležitějším, protože umožňuje volné užívání a šíření dat. V některých případech se může vyskytnout potřeba nějak způsob nakládání s daty upravit, například zavést povinnost při každém jejich použití uvést autora původních dat. Pro tyto účely lze užít některé ze sady specializovaných licencí, např. licence z populární sady Creative Commons jako Creative Commons 3.0\footnote{https://creativecommons.org/licenses/by/3.0/cz/} nebo Public domain zero.\footnote{https://creativecommons.org/publicdomain/zero/1.0/}

Míru dostupnosti a vyhledatelnosti otevřených dat na Internetu zvyšují různé specializované katalogy a indexy. Tyto umožňují procházet a vyhledávat datové sady podle různých kritérií (metadat), jako např. téma, autor, klíčové slovo, datum vydání a aktualizace, použitá licence a další. Národní katalogy otevřených dat takto poskytují přehledy otevřených datových sad daného státu. Více o konkrétních poskytovatelích dat v sekci Zdroje a poskytovatelé otevřených dat.

Požadavek na formát s volně přístupnou specifikací vyplývá ze snahy vyhnout se závislosti na software, za jehož užití při práci s otevřenými daty by musel uživatel platit. Z tohoto pohledu nejsou formáty jako XLS nebo DOC firmy Microsoft otevřené, protože jejich použití je vázáno na nutnost zakoupit příslušné programy sady Microsoft Office. Naproti tomu s formáty jako CSV, XML, RDF nebo s otevřenými verzemi Microsoftích formátů DOCX a XLSX je možné pracovat pomocí volně (zadarmo) dostupných nástrojů, a tedy jsou z pohledu definic otevřených dat považovány za otevřené.

Vnitřní struktura dat hraje roli pro možnost data strojově zpracovávat. Ačkoliv člověk je v textu schopen pohledem různým částem dokumentu (nadpis, poznámka pod čarou, seznam) přiřadit význam a odpovídajícím způsobem je interpretovat, stroj (software) toho schopen není.\footnote{Výjimkou je software zpracovávající přirozený jazyk (např. pomocí metod strojového učení), který je ale mimo kontext této práce a jehož rozpoznávací schopnosti nejsou stoprocentní.} Tento problém se řeší zavedením určitého řádu pro uložení informací, neboli vnitřní struktry dat. Taková struktura může být poměrně volná, jako u formátu CSV,\footnote{https://en.wikipedia.org/wiki/Comma-separated\_values} kde jsou data umístěna v řádcích a jednotlivá pole v řádku jsou oddělená čárkou. Naopak různé binární formáty\footnote{https://en.wikipedia.org/wiki/Binary\_file} vyžadují strukturu pevnou, kdy jsou data pevně zarovnána do bloků začínajících na pevně daných adresách v paměti, což umožňuje zpracovávajícímu software spolehnout se, že data s určitým významem jsou v paměti počítače uložena na předem definovaném místě. Možností strukturování dat, která je někde napůl cesty mezi prvními dvěma popsanými způsoby, je značkování. Například formát XML\footnote{https://en.wikipedia.org/wiki/XML} používá tzv. tagy, neboli textové značky, které ohraničují část textu a dávají mu tak význam, který umí aplikace interpretovat. Takto například  z původní, pro program nečitelné věty \textit{\uv{Jmenuji se Tomáš Beňák.}}, lze pomocí XML vyrobit větu \textit{\uv{Jmenuji se <Jmeno>Tomáš</Jmeno><Prijmeni>Beňák</Prijmeni>.}}, ze které už program, který v textu výskyt tagů \textsf{<Jmeno>} a \textsf{<Prijmeni>} očekává, může extrahovat jméno a příjmení.

\section{Zdroje a poskytovatelé otevřených dat}
% TODO tady uvést příklady otevřených dat - přehledy hospodaření, výsledky voleb, základní statistiky. (nebo v úvodu?) - definuj veřejnou správu, veřejný sektor, veřejná data
Přirozeným zdrojem otevřených dat jsou orgány veřejné správy a samosprávy (orgány veřejné moci, dále jen \uv{veřejná správa}), jako například vláda, ministerstva, obce a úřady, obecně pak i různé další nevládní a neziskové organizace (dále jen \uv{veřejný sektor}). Na řadu z informací poskytovaných veřejnou správou existuje zákonný nárok, například v České republice podle zákona 106/1999 Sb. o svobodném přístupu k informacím, a zákona 123/1998 Sb. o právu na informace o životním prostředí. Kromě toho je Česká republikace jako členský stát Evropské unie vázána směrnicí PSI\footnote{Směrnice Evropského parlamentu a Rady č. 2003/98/ES} o opakovaném použití informací veřejného sektoru, která ukládá členským zemím povinnost poskytovat veřejná data jako otevřená.

Zveřejňování dat veřejné správy bývá zastřešeno na úrovni státu portály pro otevřená data, které poskytují katalogy otevřených datových sad. V takových katalozích vydavatelé dat registrují poskytovaná data a uživatelé katalogů v nich pak otevřené datové sady na základě různých kritérií vyhledávají. V zahraničí jsou ukázkovými příklady takových projektů portály otevřených dat USA a Velké Británie.\footnote{https://www.data.gov resp. https://data.gov.uk} Otevřené datové sady veřejných institucí v České republice jsou katalogizovány v Národním katalogu otevřených dat (NKOD)\footnote{http://portal.gov.cz/portal/obcan/rejstriky/data/97898} v rámci portálu veřejné správy.

Kromě institucí veřejné správy se publikováním otevřených dat zabývají i další instituce veřejného sektoru, případně i soukromé a mezinárodní organizace. Mezi populární zahraniční poskytovatele otevřených dat patří mimo jiné britský deník Guardian\footnote{http://www.theguardian.com/data} nebo OSN\footnote{http://data.un.org} či OECD.\footnote{http://stats.oecd.org} Z těch českých pak uveďme například organizaci Opendata.cz.\footnote{http://opendata.cz/}

% Vypustit? (nedůležité)
%Kromě samotného publikování dat je důležité a přínosné i úsilí věnované propagaci a popularizaci otevřených dat, jak směrem k odborné, tak i laické veřejnosti. Například výše zmíněný projekt Open Data Index se věnuje mapování dostupných otevřených datových sad poskytovaných státními institucemi, a státy hodnotí podle množství zpřístupněných veřejných dat. V České republice projekt pro otevřená data Fondu Otakara Motejla\footnote{http://www.otevrenadata.cz} poskytuje platformu pro spolupráci odborníků,\footnote{http://www.otevrenadata.cz/o-nas/forum-pro-otevrena-data} veřejnost zase seznamuje s koncepty, přínosy a riziky otevření dat a slouží i jako rozcestník vedoucí na konkrétní otevřené datové sady a stránky poskytovatelů dat.

\section{Otevřené datové sady a aplikace}

Co přesně za data jsou poskytována poskytovateli, kteří jsou zmíněni v předchozí podkapitole?
Příklady aplikací a datasetů stačí v odrážkách a s footnotes pro ilustraci nabídky. Důležité je zaměřit se na data a aplikace o kriminalitě.

%Data poskytovaná veřejnou správou a samosprávou typicky zahrnují např. informace o výdajích státu, veřejných rozpočtech, výběrových řízeních na veřejné zakázky, výsledky voleb, přehled legislativy nebo základní statistické ukazatele pro danou krajinu. Open knowledge foundation zveřejnila v rámci projektu Open Data Index\footnote{http://index.okfn.org} seznam datových sad veřejného sektoru doporučených ke zveřejnění.

V současnosti existuje již poměrně mnoho otevřených datových sad, nabízejících informace z nejrůznějších odvětví lidské činnosti. Data jsou publikována v různých otevřených formátech, přičemž publikace pomocí RDF jako Linked data je stále poměrně málo častá. Nejběžnějšími formáty jsou CSV a XML.

Veřejné instituce po celém světě zpravidla zveřejňují informace o svém hospodaření.

Obecně jsou velice populární data vázaná geograficky, místně. Nad takovými sadami rostou například mobilní aplikace, které nabízejí uživateli informace relevantní pro jeho aktuální polohu. Příkladem takové aplikace je např. AirNOW, která uživatele informuje o aktuální kvalitě ovzduší v daném místě. Další takovou aplikací je Alternative Fueling Station Locator, který uživateli našeptává polohu nejbližších benzínek. Takovýchto aplikací je velká spousta, některé jsou poměrně jednoduchými aplikacemi geolokalizace informací na mapovém či jiném podkladě, jiné jsou i značně sofistikované, například Climate FieldView, aplikace, která farmářům umožňuje na základě monitorování a předpovědí lokálního počasí optimalizovat rozhodnutí o hospodaření.
Zajímavým odvětvím aplikací jsou různé hlásiče pro nespokojence. Občan USA může pomocí aplikace Civic Request Tracker reportovat místním autoritám problém, na který narazil a který by měly tyto autority řešit. Obdobná aplikace funguje i ve Velké Británii, kde lidé mohou hlásit díry a nerovnosti v silnicích a chodnících.
Aplikace nad otevřenými daty mohou podporovat informované rozhodování občanů v různých oblastech. College Affordability and Transparency Center budoucím studentům a jejich rodičům umožňuje poskytovat informace zejména o nákladech spojených se vzděláváním na vysokých školách a univerzitách.
Jiné aplikace mohou býz zdrojem různých demografických přehledů, srovnání i pouhých zajímavostí. City Data zastřešuje velké množství datových sad popisujících nejrůznější fakta o městech USA, které srovnává v řadě zajímavých žebříčků.

V České republice je dobrým rozcestníkem aplikací web Fondu Otakara Motejla otevrenadata.cz. Každoročně se do soutěže Společně otevíráme data přihlásí řádově desítky aplikací, ze kterých pak porotci vyberou ty nejlepší. Tři proběhlé ročníky ukázaly velice zajímavé a užitečné aplikace, jakou je například Léková encyklopedie, umožňující lékařům snadno nacházet kontraindikace léků. Aplikace postavené nad vlastními datasety nabízí i Opendata.cz\footnote{http://www.opendata.cz/cs/node/21}, z nich nejzajímavější jsou Mapa veřejných zakázek a Hospodaření obcí.

Nad otevřenými daty dosud vzniklo mnoho zajímavých a užitečných aplikací jak ve světě, tak i v ČR. Z těch zahraničních jmenujme například TheyWorkForYou\footnote{http://www.TheyWorkForYou.com}, zaznamenávající hlasování členů britského parlamentu, WhereDoesMyMoneyGo \footnote{http://www.wheredoesmymoneygo.org/}, poskytující přehled o využití daní britských daňových poplatníků, nebo celé sady aplikací od MySociety\footnote{https://www.mysociety.org/projects/} a od Sunlight Foundation\footnote{http://sunlightfoundation.com/tools/}.

\section{Rizika publikování otevřených dat}

Publikování dat v otevřené podobě přináší řadu výhod, popsaných 
\begin{itemize}
	\item data lze snadno dohledat na internetu, s minimem nákladů (v drtivé většině případů bezplatně) si je stáhnout, volně s nimi nakládat a dále je s minimem omezení (třeba jen s uvedením jejich autora) šířit
	\item s daty lze pracovat pomocí volně dostupných nástrojů => šetří peníze, které už jsme jednou (u veřejných dat) vydali
	\item centralizovaná řešení s národními katalogy umožňují snadno nelézt data na jednom místě
	\item strojová zpracovatelnost usnadňuje manipulaci dat programy a tak umožňuje s daty snadno zacházet jako se surovinou, tedy je analyzovat, vizualizovat, nebo nad nimi stavět aplikace
	\item nad daty vzniká spousta užitečných aplikací (viz výše)
	\item pravidelná publikace dat může veřejné správě ušetřit práci s obsluhou požadavků na poskytnutí informací dle zákonů o svobodném přístupu k informacím (viz výše), udělat si pořádek ve vlastních datech a procesech, které je vytvářejí.
\end{itemize}

Kromě těchto pozitivních efektů obnáší publikace dat v otevřené formě (zejména dat veřejné správy) jistá rizika, která je nutné analyzovat a řídit:
\begin{itemize}
	\item nelze v otevřené podobě zveřejňovat osobní a jiné citlivé (např. podléhající utajení) údaje. Tento problém lze obecně řešit agregací\footnote{https://en.wikipedia.org/wiki/Aggregate\_data} nebo anonymizací dat\footnote{https://en.wikipedia.org/wiki/Data\_anonymization}
	\item publikovaná data je nutné dobře popsat, aby nemohlo dojít k jejich mylné interpretaci, která by mohla poškodit poskytovatele dat
	\item s vydáváním dat v otevřené podobě je spojená určitá režie, jednak počáteční náklady spojené s vytvořením publikačního plánu...
\end{itemize}

\subsection{Otevřená data o kriminalitě a jejich využití}

Dříve, než začneme popisovat práci na nové datové sadě o kriminalitě, musíme se seznámit se základními koncepty sémantického webu, relevantními technologiemi a dostupnými softwarovými nástroji, které nám umožní efektivní postup práce při návrhu a vytvoření datasetu i ukázkové aplikace.