% !TeX spellcheck = cs_CZ
\chapter*{Úvod}
\addcontentsline{toc}{chapter}{Úvod}
Veřejný sektor v České republice vyprodukuje každoročně ohromné množství dat. Jedná se o nejrůznější statistiky, analýzy, smlouvy, strategické dokumenty, zákony nebo informace o hospodaření. Část těchto dat slouží pouze interním potřebám institucí, část ani zveřejňovat nelze, protože takovému zveřejnění brání legislativní překážky. Existuje však nezanedbatelná masa informací, které lze zpřístupnit veřejnosti.

V posledních deseti letech získalo zpřístupňování dat veřejného správy na popularitě a společné úsilí různých neziskových informací a iniciativ, akademické sféry i státní správy a samosprávy dalo vznik konceptu tzv. otevřených dat (open data)\footnote{TODO}. Otevřená data lze za vhodně nastavených podmínek užití dále zhodnocovat, ať už v komerční či akademické sféře, různými nevládními a neziskovými organizacemi či nadšenci z řad jednotlivců. Tato data pak slouží jako surovina pro vytváření hodnotných informačních produktů, jakými jsou například různé počítačové a mobilní aplikace, přinášející ekonomický zisk pro firmy nebo zkvalitnění služeb pro veřejnost.

Rozvoj technologií sémantického webu\footnote{TODO} přinesl nové možnosti, jak publikovat data na globální úrovni. Tim Berners-Lee, otec World Wide Web, definoval koncept tzv. propojených dat, tzv. Linked Data, zavádějící sadu principů\footnote{TODO}, podle kterých data publikovat způsobem, který dramaticky zvyšuje jejich dostupnost [...TODO kecy]. S využitím existujících standardů a technologií webu dokumentů (HTTP, URI, XML) a technologií webu sémantického (RDF, SPARQL) jsou data publikována způsobem, umožňujícím jejich vzájemné propojení na celosvětové úrovni, jako je tomu u stávajícího webu hypertextových dokumentů. Filozofie otevřených dat je s filozofií dat propojených kompatibilní a je možné data otevřená publikovat jako otevřená propojená data (Linked Open Data\footnote{TODO}), což jen dále zvyšuje jejich globální upotřebitelnost.

Cílem této práce je vytvořit pravidelně aktualizovanou datovou sadu s údaji o kriminalitě, vztahujícími se k České republice, a tuto sadu publikovat jako otevřená propojená data podle principů Linked Data s využitím relevantních technologií sémantického webu. Užitečnost datové sady bude demonstrována ukázkovou aplikací, vyvinutou v rámci této práce.

Zbývající část práce je členěna následujícím způsobem. Kapitola 2 čtenáře podrobněji uvádí do problematiky otevřených dat a přináší stručný přehled zajímavých otevřených datových sad a aplikací postavených nad otevřenými daty. Kapitola 3 představuje otevřená data v kontextu sémantického webu a přináší seznámení s relevantními koncepty a technologiemi, kapitola 4 pak třetí kapitolu doplňuje o přehled souvisejících technologií a software pro práci s tzv. prostorovými daty (geodata). Pátá kapitola přesně vymezuje cíle práce a specifikuje konkrétní požadavky na výsledné řešení. Kapitola šestá se zabývá analýzou a výběrem dostupných zdrojů dat o kriminalitě, sedmá kapitola pak řeší návrh verzovacího schematu pro datovou sadu. Osmá kapitola popisuje návrh datového modelu datové sady a návrh ontologie pro popis faktů o kriminalitě. Kapitola 9 pak provádí čtenáře procesem extrakce dat z vybraných zdrojů do RDF, a následující kapitola 10 pak popisuje logickou organizaci datové sady do RDF datasetů a také publikaci datové sady v RDF store. Kapitola 11 prochází proces napojení datové sady na další datové sady z globálního prostoru propojených dat (Linked data). Kapitola 12 popisuje specifikaci požadavků, návrh a implementaci aplikace, která užitečnost datové sady demonstruje prakticky. Závěrečná třináctá kapitola přináší zhodnocení dosažených výsledků a nastiňuje směry, kterými by se mohl ubírat další rozvoj datové sady a ukázkové aplikace.