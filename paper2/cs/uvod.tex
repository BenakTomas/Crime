\chapter*{Úvod}
\addcontentsline{toc}{chapter}{Úvod}

% [Data o trestné činnosti a jejich využití]
Trestná činnost vždy byla, je a patrně stále bude populárním tématem jak pro laickou, tak i pro odbornou veřejnost. Zprávy černé kroniky patří k těm nejpopulárnějším, snad pro jistou morbidní zvědavost, která je lidem tak vlastní. Kromě tohoto, řekněmě populárního využití dat o trestné činnosti, existují i na první pohled nudnější, o to však důlěžitější a smysluplnější způsoby využití, a to sice ve formě kriminálních statistik.

% [Kriminální statistiky]
Kriminální statistiky, nebo také statistiky zločinnosti, jsou důležitým zdrojem informací o společnosti. Zrcadlí se v nich její neduhy a zprostředkovaně odrážejí sociální a ekonomickou situaci. Statistickou analýzou těchto dat je možné sledovat trendy, odhadovat vlivy různých veličin na páchání trestné činnosti, objevovat dosud nepopsané korelace. Výstupem takové činnosti pak může být lepší (efektivnější, přesnější, cílenější) práce bezpečnostních orgánů státu, zejména policie. Ostatní státní instituce, jako například Ministerstvo práce a sociálních věcí, pak mohou v problematických oblastech řešit problém intenzivněji.

Zdrojem dat pro tyto statistiky je Policie České republiky (PČR). V rámci své činnosti (TODO ze zákona?) PČR shromažďuje velké množství dat o zjištěné trestné činnosti a pachatelích. Tato data PČR upotřebí jednak pro vlastní taktické a strategické účely, druhak také na jejich základě připravuje statistické sestavy, které ze zákona publikuje\footnote{http://www.policie.cz/statistiky-kriminalita.aspx}. Data jsou poskytována ve formě excelových souborů, které jsou volně ke stažení. Plusem je jejich přehlednost pro čtenáře a také stálost vnitřní struktury těchto souborů, která se již po léta nemění. Nevýhodou je, že ekvivalentní data v neexistují v některém ze standardních formátů používaných pro strojovou výměnu a zpracování dat, jako např. XML. Zatímco excel (alespoň ve verzi používané ve statistických sestavách) je formátem proprietárním a pro výměnu dat standardně nepoužívaným, XML je v tomto ohledu již dobře zavedeným a respektovaným formátem. Již léta spolu aplikace komunikují v řeči různých typů XML dokumentů. Možnost definovat formálně strukturu XML dokumentu pomocí šablony umožňuje strojovou kontrolu správné struktury dat.

- jak je to v UK, zdroje